\documentclass[a4paper]{amsart}
\usepackage[leqno]{amsmath}
\usepackage{amssymb}
\usepackage{amscd}
\usepackage{amsthm}
\usepackage{mathrsfs}
\usepackage{mathtools}
\usepackage{bbm}
\usepackage{color}
\usepackage{enumerate}
\usepackage{cite}
\usepackage[utf8]{inputenc}
\usepackage[all,cmtip]{xy}
\usepackage{etoolbox}
\usepackage{tikz}
\usepackage{extarrows}

\numberwithin{equation}{section}

\newcommand{\N}{\ensuremath{\mathbb{N}}}							
\newcommand{\Z}{\ensuremath{\mathbb{Z}}}
\newcommand{\Q}{\ensuremath{\mathbb{Q}}}
\newcommand{\R}{\ensuremath{\mathbb{R}}}
\newcommand{\C}{\ensuremath{\mathbb{C}}}
\newcommand{\F}{\ensuremath{\mathbb{F}}}

\newcommand{\A}{\ensuremath{\mathbb{A}}}
\newcommand{\I}{\ensuremath{\mathbb{I}}}
\newcommand{\T}{\ensuremath{\mathbb{T}}}


\DeclareMathOperator{\Hom}{Hom}
\DeclareMathOperator{\End}{End}
\DeclareMathOperator{\Ext}{Ext}
\DeclareMathOperator{\Aut}{Aut}
\DeclareMathOperator{\Gal}{Gal}
\DeclareMathOperator{\Sym}{Sym}

\DeclareMathOperator{\Ker}{ker}
\DeclareMathOperator{\Koker}{coker}
\DeclareMathOperator{\im}{im}
\DeclareMathOperator{\rk}{rk}

\DeclareMathOperator{\Rea}{Re}
\DeclareMathOperator{\Ima}{Im}

\DeclareMathOperator{\id}{id}
\DeclareMathOperator{\chr}{char}


\DeclareMathOperator{\ord}{ord}
\DeclareMathOperator{\Ind}{\ensuremath{Ind}}
\DeclareMathOperator{\Dist}{Dist}
\DeclareMathOperator{\St}{St}
\DeclareMathOperator{\Br}{Br}
\DeclareMathOperator{\cind}{c-ind}
\DeclareMathOperator{\Coind}{Coind}
\DeclareMathOperator{\sm}{sm}
\DeclareMathOperator{\sing}{sing}
\DeclareMathOperator{\fd}{fd}
\DeclareMathOperator{\hol}{hol}
\DeclareMathOperator{\tr}{tr}
\DeclareMathOperator{\tor}{tor}
\DeclareMathOperator{\MT}{MT}
\DeclareMathOperator{\cyc}{cyc}
\DeclareMathOperator{\anti}{anti}
\DeclareMathOperator{\HH}{H}
\DeclareMathOperator{\MS}{\mathcal{M}}
\renewcommand{\det}{\operatorname{det}}
\newcommand{\CG}{\hspace{1em}~^C G}
\newcommand{\cf}{{\mathbbm 1}}
\newcommand{\PP}{\ensuremath{\mathbb{P}^1}}
\newcommand{\II}{\ensuremath{\mathbb{I}}}	
\newcommand{\VV}{\ensuremath{\mathbb{V}}}	
\newcommand{\WW}{\ensuremath{\mathbb{W}}}	
\newcommand{\sinfty}{\ensuremath{^{S,\infty}}}			
\newcommand{\pinfty}{\ensuremath{^{\p,\infty}}}
\newcommand{\Ah}{\ensuremath{\mathcal{A}}}
\DeclareMathOperator{\rec}{rec}
\DeclareMathOperator{\Div}{Div}
\DeclareMathOperator{\cont}{ct}
\DeclareMathOperator{\an}{an}
\DeclareMathOperator{\mx}{max}
\DeclareMathOperator{\rig}{rig}
\DeclareMathOperator{\loc}{loc}
\DeclareMathOperator{\al}{al}
\DeclareMathOperator{\univ}{univ}
\newcommand{\n}{\ensuremath{\mathfrak{n}}}
\newcommand{\m}{\ensuremath{\mathfrak{m}}}
\newcommand{\p}{\ensuremath{\mathfrak{p}}}
\newcommand{\q}{\ensuremath{\mathfrak{q}}}
\newcommand{\f}{\ensuremath{\mathfrak{f}}}
\newcommand{\LI}{\mathcal{L}}
\newcommand{\No}{\ensuremath{N}}
\newcommand{\Tr}{\ensuremath{Tr}}
\DeclareMathOperator{\pr}{pr}
\newcommand{\Hec}{\mathscr{H}}
\DeclareMathOperator{\ev}{ev}
\DeclareMathOperator{\ob}{ob}



\newcommand{\into}{\hookrightarrow}
\newcommand{\onto}{\twoheadrightarrow}
\newcommand{\too}{\longrightarrow}								%Arrows
\newcommand{\mapstoo}{\longmapsto}

\newtheorem{Lem}{Lemma}[section]
\makeatletter
\newlength{\@thlabel@width}%
\newcommand{\thmenumhspace}{\settowidth{\@thlabel@width}{\itshape1.}\sbox{\@labels}{\unhbox\@labels\hspace{\dimexpr-\leftmargin+\labelsep+\@thlabel@width-\itemindent}}}
\makeatother
\newtheorem{Pro}[Lem]{Proposition}
\newtheorem{MLem}[Lem]{Main Lemma}
\newtheorem{Thm}[Lem]{Theorem}
\newtheorem{DefLem}[Lem]{Definition/Lemma}
\newtheorem{Def}[Lem]{Definition}
\newtheorem{Rem}[Lem]{Remark}
\newtheorem{Con}{Conjecture}
\renewcommand*{\theCon}{\Alph{Con}}
\newtheorem{Ass}{Assumption}
\renewcommand*{\theAss}{\Alph{Ass}}
\newtheorem{Cor}[Lem]{Corollary}
\newtheorem{Exa}[Lem]{Example}
\newtheorem*{Hyp}{Hypothesis}

\DeclarePairedDelimiter{\floor}{\lfloor}{\rfloor}

\author[L. Gehrmann]{Lennart Gehrmann}
\address{L. Gehrmann \\ Fakult\"at f\"ur Mathematik \\ Universit\"at Duisburg-Essen \\ Thea-Leymann-Stra\ss e 9 \\ 45127 Essen \\ Germany}
\email{lennart.gehrmann@uni-due.de}

\title[Stark-Heegner cycles]{Stark-Heegner cycles over arbitrary number fields}
\subjclass[2010]{}


\setcounter{tocdepth}{1}

\begin{document}


\begin{abstract}
TODO
\end{abstract}

\maketitle


\tableofcontents

\section*{Introduction}

TODO

\smallskip
\textbf{Notations.}
All rings are assumed to be commutative and unital.
The group of invertible elements of a ring $R$ will be denoted by $R^{\ast}$.
Given an $R$-module $M$ we put $M^{\vee}=\Hom_R(M,R)$.
If $S$ is an $R$-algebra and $M$ an $R$-module, we put $M_S=M\otimes_R S.$
If $R$ is a ring and $G$ a group, we will denote the group ring of $G$ over $R$ by $R[G]$.
Let $H$ be an open subgroup of a locally profinite group $G$ and $M$ an $R$-linear representation $M$ of $H$.
The \textit{compact induction} $\cind^{G}_{H}M$ of $M$ from $H$ to $G$ is the space of all functions $f\colon G\to M$ such that:
\begin{itemize}
\item $f$ has finite support modulo $H$ and
\item $f(hg)=h.f(g)$ for all $h\in H, g\in G$.
\end{itemize}
Compact induction $\cind^{G}_{H}M$ is an $R$-module on which $G$ acts $R$-linearly via the right regular representation.
Let $\chi\colon G\to R^{\ast}$ be a character.
We write $R[\chi]$ for the $G$-representation, which underlying $R$-module is $R$ itself and on which $G$ acts via the character $\chi$.
More generally, if $M$ is any $R[G]$-module, we put $M(\chi)=M\otimes_R R(\chi).$
The trivial character will be denoted by $\cf$.

\smallskip
\textbf{Acknowledgements.}
TODO
\section{The setup}\label{Setup}
We fix an algebraic number field $F$ with ring of integers $\mathcal{O}$.
In addition, we fix a finite place $\p$ of $F$ lying above the rational prime $p$ and choose embeddings
$$\C \xhookleftarrow{\iota_{\infty}} \overline{\Q} \xhookrightarrow{\iota_{p}} \overline{\Q_p}.$$
We let $\Sigma$ denote the set of all embeddings $\sigma\colon F\into \C$ and for a prime $v$ lying above $p$ we let $\Sigma_v$ be the set of all continuous embeddings $\Sigma_{v}$.
The two chosen embeddings $\iota_{\infty}$ and $\iota_{p}$ yield a decomposition
$$\Sigma=\bigcup_{v\mid p}\Sigma_v.$$

We denote the number of real places of $F$ by $r_\R$ and the number of complex places by $r_\C$.
If $v$ is a place of $F$, we denote by $F_{v}$ the completion of $F$ at $v$.
If $v$ is a finite place, we let $\mathcal{O}_{v}$ denote the valuation ring of $F_{v}$ and $\ord_{v}$ the additive valuation such that $\ord_{v}(\varpi)=1$ for any local uniformizer $\varpi\in\mathcal{O}_{v}$.
We write $\mathcal{N}(v)$ for the cardinality of the residue field of $\mathcal{O}_{v}$.

Let $\A$ be the adele ring of $F$, i.e~the restricted product over all completions $F_{v}$ of $F$.
We write $\A^\infty$ (respectively $\A\pinfty$) for the restricted product over all completions of $F$ at finite places (respectively finite places different from $\p$).
More generally, if $S$ is a finite set of places of $F$ we denote by $\A^{S}$ the restricted product of all completions $F_v$ with $v\notin S$.

If $H$ is an algebraic group over $F$ and $v$ is a place of $F$, we write $H_v=H(F_v)$.
If $l$ is a (possible infinite) rational place we put $H_l=\prod_{v\mid l}H_v$.
Further, we put $H_p^{\p}=\prod_{v\mid p,\ v\neq \p}H_v.$

Throughout the article, we fix an inner form $\widetilde{G}$ of the algebraic group $GL_2/F$, which is split at the prime $\p$.
We denote the centre of $\widetilde{G}$ by $Z$ and put $G=\widetilde{G}/Z$.
If $G$ is split, we always identify it with $PGL_2$.
Similarly, if $v$ is a place of $F$ at which $G$ is split, we choose an isomorphism of $G_v$ with $PGL_2(F_v)$.
We write $q$ for the number of Archimedean places at which $G$ is split.


At last, we fix a cuspidal automorphic representation $\pi=\otimes_v \pi_v$ of $G(\A)$ with the following properties:
\begin{itemize}
\item $\pi$ is cohomological with respect to an algebraic coefficient system $V_{\al,\C}$ (see Section \ref{Component} for more details)
 and
\item $\pi_\p$ is the (smooth) Steinberg representation $\St^{\infty}_{\p}(\C)$ of $G_\p=PGL_2(F_\p)$.
\end{itemize}




\section{Automorphic L-invariants}
The aim of this section is to define automorphic $\LI$-invariants.
We follow broadly the same steps as Spie\ss~in the Hilbert modular parallel weight $2$ setting (see \cite{Sp}), though our arguments are more involved since our representation is not necessarily ordinary anymore.
\subsection{Cohomology of $\p$-arithmetic groups}\label{Cohomology}
Throughout this section we fix a ring $R$.

Let $\Div(\PP(F))$ denote the free abelian group on $\PP(F)$ and $\Div_0(\PP(F))$ the kernel of the map $$\Div(\PP(F)) \to \Z,\ \sum_P m_P P \mapsto \sum_P m_P.$$
The $PGL_2(F)$-action on $\PP(F)$ induces an action on $\Div_0(\PP(F))$.
If $G$ is non-split, we put $\HH^{i}_c(G(F),A)=\HH^{i}(G(F),A)$.
If $G$ is split, we define $\HH^{i}_c(G(F),A)=\HH^{i-1}_c(G(F),\Hom_\Z(\Div_0(\PP(F)),A)).$
In this case the boundary map associated to the short exact sequence
$$0\too A \too\Hom_\Z(\Div(\PP(F)),A)\too\Hom_\Z(\Div_0(\PP(F)),A)\too 0$$
yields a map 
\begin{align}\label{delta}
\delta\colon\HH^{i}_c(G(F),A)\too\HH^{i}(G(F),A).
\end{align}

Given a compact, open subgroup $K^\p\subseteq G(\A\pinfty)$, an $R[K^{\p}]$-module $N^{\p}$, an $R[G_\p]$-module $M_\p$ and an $R[G(F)]$-module $N$ we define $\Ah_{R}(K^{\p},N^{\p},M_\p;N)$ as the space of all $R$-bilinear maps $\Phi\colon G(\A\pinfty) \times N^{\p} \times M_\p\to N$
such that $\Phi(gk,kn,m)=k\Phi(g,n,m)$ for all $g\in  G(\A\pinfty)$, $k\in K^\p$, $n\in\N^{\p}$ and $m\in M_\p$.
The $R$-module $\Ah_R(K^\p,N^\p,M_\p;N)$ carries a natural $G(F)$-action given by
$$(\gamma.\Phi)(g,n,m)=\gamma.(\Phi(\gamma^{-1}g,n,\gamma^{-1}.m)).$$
Most of the times the module $N^{\p}$ is equal to $R$.
In this case we put $$\Ah_R(K^\p,M_\p;N)=\Ah_R(K^\p,R,M_\p;N).$$

\begin{Exa}\label{Example}
If $M_\p$ is of the form $\cind_{K_\p}^{G_\p} R$ for some compact, open subgroup $K_\p\subseteq G_\p$, we put
$$\Ah(K^\p K_\p;N)=\Ah_{R}(K^\p,M_\p;N)$$
where $?\in\left\{\emptyset,c \right\}$.
By definition we have a natural $G(F)$-equivariant isomorphism
$$\Ah(K^\p K_\p;N)\xrightarrow{\cong}C(G(\A^\infty)/K^\p K_\p,N).$$

More generally, suppose $M_\p$ is of the form $\cind_{K_\p}^{G_\p} N_\p$ for some compact, open subgroup $K_\p\subseteq G_\p$ and some $R[G_\p]$-module $N_\p$
and that $N^{\p}$ is a $G(\A\pinfty)$-module.
Then the map
$$\left(\cind_{K_\p}^{G_\p} R\right) \otimes_R N_p \too \cind_{K_\p}^{G_\p} N_\p,\ (f,n)\mapstoo [g \mapsto f(g)\cdot g.n]$$
is an isomorphism of $R[G_\p]$-modules.
Hence, its inverse (and a similar map for the $N^{\p}$-part) induces an isomorphism of $R[G(F)]$-modules
$$\Ah_{R}(K^\p,N^{\p},M_\p;N)\xrightarrow{\cong}C(G(\A^\infty)/K^\p K_\p,\Hom_R(N^{\p}\otimes_R N_\p, N)).$$
\end{Exa}

\begin{Def}
An $R[G_\p]$-module $M$ is called flawless if 
\begin{itemize}
\item $M$ is projective as an $R$-module and
\item there exists a finite length exact resolution
$$0\too P_m\too\cdots\too P_0\too M \too 0$$
of $R[G_\p]$-modules, where each $P_i$ is a finite direct sum of modules of the form
$$\cind_{K_\p}^{G_\p} L$$
with $K_\p\subseteq G_\p$ a compact, open subgroup and $L$ an $R[K_\p]$-module which is finitely generated projective over $R$.
\end{itemize}
\end{Def}

\begin{Pro}\label{FlachundNoethersch}
Suppose that $M$ is a flawless $R[G_\p]$-module and that $N^{\p}$ if finitely generated projective as an $R$-module.
For $?\in\left\{\emptyset,c \right\}$ we have:
\begin{enumerate}[(a)]
\item\label{FuN} The $R$-module $\HH^{d}_?(G(F),\Ah_{R}(K^\p,N^{\p},M_\p;N))$ is finitely generated for all $d$ if $R$ is Noetherian and $N$ is finitely generated as an $R$-module.
\item\label{FN2} If $S$ is a flat $R$-algebra, then the canonical map
	 \begin{align*}
	\HH^{d}_?(G(F),\Ah_{R}(K^\p,N^{\p},M_\p;N))\otimes_R S
	\too &\HH^{d}_?(G(F),\Ah_{S}(K^\p,N^{\p}_S, M_{\p,S};N_S))
	\end{align*}
	is an isomorphism for all $d\in\Z$.
\end{enumerate}
\end{Pro}
\begin{proof}
This is essentially Proposition 4.9 of \cite{Ge}.
\end{proof}

\begin{Exa}
Let $N$ be an $R[G(F)]$-module and $K^{\p}K_\p\subseteq G(\A^{\infty})$ a compact, open subgroup.
In light of Example \ref{Example} we put
\begin{align*}
\HH^{d}(X_{K^{\p}K_\p},N)&=\HH^{d}(G(F),\Ah(K^\p K_\p;N))
\intertext{respectively}
\HH^{d}_c(X_{K^{\p}K_\p},N)&=\HH^{d}_c(G(F),\Ah(K^\p K_\p;N)).
\end{align*}
If $K^\p K_\p$ is neat or $R$ is a field of characteristic $0$, we can identify these groups with the $N$-valued singular cohomology (respectively singular cohomology with compact support) of the locally symmetric space of level $K^\p K_\p$ associated to $G$.
\end{Exa}

Let $\Omega$ be a finite extension of $\Q_p$ with ring of integers $R$, $V_\p$ an $\Omega$-Banach representation of $G_\p$ and $V^{\p}$ a finite dimensional continuous $\Omega$-representation of $G_p^{\p}$.
We view $V^{\p}$ as a $G(F)$-representation via the embedding $G(F)\into G_p^{\p}$.
Let $\epsilon\colon \pi_0(G_\infty)\to \left\{\pm 1\right\}$ be a sign character.
We define
\begin{align*}
\Ah_\Omega^{\cont}(K^\p,V_\p;V^{\p}(\epsilon))&=C(G(\A\pinfty)/K^\p,\Hom_{\Omega,\cont}(V_\p,V^{\p}(\epsilon))).
\end{align*}

Now let $V_\p$ merely be a be a locally convex topological $\Omega$-vector space equipped with a continuous $G_\p$-action.
Suppose that $V_\p$ admits an open $R[G_\p]$-lattice $M_\p$ that is flawless.
Since $M_\p$ is finitely generated, it follows that the completion of $V_\p$ with respect to $M_\p$ is the universal unitary completion $V_\p^{\univ}$ of $V_\p$.
We have the following automatic continuity statement.
\begin{Cor}\label{automatic}
Let $V_\p$ be a finite length, locally $\Q_p$-algebraic representation of $G_\p$ that admits a $G_\p$-stable separated $R$-lattice and let $V^{\p}$ be a finite dimensional $\Omega$-representation of $G_p^{\p}$.
Then the canonical map
$$\HH^{d}_?(G(F),\Ah_{\Omega}^{\cont}(K^\p,V_\p^{\univ};V^{\p}(\epsilon)))\too \HH^{d}_?(G(F),\Ah_{\Omega}(K^\p,V_\p;V^{\p}(\epsilon)))$$
is an isomorphism for all characters $\epsilon$ and $?\in\left\{\emptyset,c\right\}$.
\end{Cor}
\begin{proof}
By \cite{Vi}, Proposition 0.4, the representation $V_\p$ admits a flawless $R$-lattice $M_\p$. 
Since $V^\p$ is finite dimensional, Example \ref{Example} implies that
$$\Ah_{\Omega}(K^\p,V_\p;V^{\p}(\epsilon))=\Ah_{\Omega}(K^\p,V^{\p,\vee},V_\p;\Omega(\epsilon)).$$
Again, by finite-dimensionality of $V^{\p,\vee}$ we see that it admits a $K_\p$-stable lattice $N_\p$.
Therefore, Proposition \ref{FlachundNoethersch} \eqref{FN2} implies that the canonical map
$$\HH^{d}_?(G(F),\Ah_{R}(K^\p,N^{\p},M_\p;R(\epsilon)))\otimes_R \Omega\too \HH^{d}_?(\Ah_{\Omega}(K^\p,V_\p;V^{\p}(\epsilon)))$$
is an isomorphism.
But the former can be identified with the cohomology group $\HH^{d}_?(G(F),\Ah_{\Omega}^{\cont}(K^\p,V_\p^{\univ};V^{\p}(\epsilon)))$
and, thus, the claim follows.
\end{proof}



\subsection{The $\pi$-isotypical component}\label{Component}
We determine the $\pi$-isotypical component of various cohomology groups.

By assumption $\pi$ is cohomological with respect to an algebraic coefficient system $V_{\al,\C}$, i.e.~there exists an irreducible algebraic $\C$-representation $V_{\sigma,\C}$ of $G_\C$ for every embedding $\sigma\in\Sigma$ such that 
$$V_{\al,\C}=\bigotimes_{\sigma\in\Sigma} V_{\sigma,\C}$$
and
$$\Hom_{\C[G(\A^\infty)]}(\pi^{\infty},\varinjlim_{K^{\p}K_\p} \HH^{\ast}(X_{K^{\p}K_\p},V_{\al,\C}^\vee))\neq 0.$$
Here we let $G(F)$ act on $V_{\sigma,\C}^\vee$ via the embedding $\sigma$.

For the remainder of the article we fix a finite extension $\Q_\pi\subseteq\overline{\Q}$ of $\Q$ such that
\begin{itemize}
\item $\left|\Hom(F,\Q_\pi)\right|=\left|\Hom(F,\overline{\Q})\right|$ and
\item the finite part $\pi\pinfty$ away from $\p$ of $\pi$ has a model over $\Q_\pi$, i.e.~$\pi\pinfty=\pi\pinfty_{\Q_\pi}\otimes_{\Q_\pi}\C.$
\end{itemize}
By the first assumption on $\Q_\pi$ each $V_{\sigma,\C}$ (viewed as an representation of $G(F)$) has a model $V_{\sigma,\Q_\pi}$ over $\Q_\pi$ and we put $V_{\al,\Q_pi}=\otimes_{\sigma}V_{\sigma,\Q_\pi}.$

Let $\Omega$ be a field extension of $\Q_\pi$ and $K^\p\subseteq G(\A\pinfty)$ a compact, open subgroup such that $(\pi\pinfty_{\Q_\pi})^{K^\p}\neq 0$.
We denote the $\Omega$-valued Hecke algebra of level $K^\p$ away from $\p$ by
$$\mathbb{\T}=\mathbb{T}(K^\p)_\Omega=C_c(K^\p\backslash G(\A\pinfty)/K^\p,\Omega).$$
If $V$ is a $\mathbb{T}(K^\p)_\Omega$-module, we write
$$V[\pi]=\Hom_{\mathbb{T}}((\pi\pinfty_\Omega)^{K^\p},V).$$

The $\Omega$-valued smooth Steinberg representation $\St_{\p,\Omega}$ of $G_\p$ is the space of all locally constant $\Omega$-valued functions on $\PP(F_\p)$ modulo constant function.
The invariants of $\St_{\p,\Omega}$ under the Iwahori subgroup $\I_\p\subseteq G_\p$ are one-dimensional.
Thus, by Frobenius reciprocity there exists a unique (up to scalar) non-zero $G_\p$-equivariant map
$$\cind_{I_\p}^{G_\p}\Omega \too \St_{\p,\Omega},$$
which in turn induces a Hecke-equivariant map
\begin{align}\label{evaluation}
\ev^{(d)}\colon \HH^{d}_?(G(F),\Ah(K^\p,\St_{\p,\Omega};N))\too \HH^{d}_?(X_{K^\p I_\p},N)
\end{align}
for every $\Omega[G(F)]$-module $N$.


\begin{Pro}\label{dimensions}
The following holds:
\begin{enumerate}[(a)]
\item\label{firstdim} For every character $\epsilon\colon \pi_0(G_\infty)\to\left\{\pm 1 \right\}$ and $?\in\left\{\emptyset,c\right\}$ we have
 $$\dim_\Omega \HH^{d}_?(X_{K^\p I_\p},V_{\al,\Omega}^\vee(\epsilon))[\pi]= \binom{r_\C}{d-q}.$$
\item\label{seconddim} The map $\ev^{(d)}$ induces an isomorphism
$$\HH^{d}_?(G(F),\Ah_\Omega(K^\p,\St_{\p,\Omega};V_{\al,\Omega}^\vee(\epsilon)))[\pi]\xrightarrow{\ev^{(d)}} \HH^{d}_?(X_{K^\p I_\p},V_{\al,\Omega}^\vee(\epsilon))[\pi]$$
for every character $\epsilon\colon \pi_0(G_\infty)\to\left\{\pm 1 \right\}$ and all $d$.
\item\label{thirddim} For every character $\epsilon\colon \pi_0(G_\infty)\to\left\{\pm 1 \right\}$ we have
$$\dim_\Omega \HH^{d}_?(G(F),\Ah_\Omega(K^\p,\St_{\p,\Omega};V_{\al,\Omega}^\vee(\epsilon)))[\pi] = \binom{r_\C}{d-q}.$$
\end{enumerate}
\end{Pro}
\begin{proof}
The proof of \cite{Ge3}, Proposition 3.7, also works in this more general setup.
\end{proof}


It is well known that the space of smooth extensions of the trivial representation $\Omega$ with the Steinberg representation is one-dimensional (see for example \cite{Cass}, Theorem 2 (b) for the case $\Omega=\C$).
We fix a smooth non-split extension
$$0\too \St_{\p,\Omega} \too \mathcal{E} \too \Omega \too 0.$$
This induces a short exact sequence
$$0\too \Ah(K^\p,\Omega;V_{\al,\Omega}^\vee(\epsilon)) \too \Ah_\Q(K,\mathcal{E};V_{\al,\Omega}^\vee(\epsilon))\too \Ah(K^\p,\St_{\p,\Omega};V_{\al,\Omega}^\vee(\epsilon)) \to 0.$$
The boundary map of the associated the long exact cohomology sequence induces the map
$$\HH^{d}_?(G(F),\Ah(K^\p,\St_{\p,\Omega};V_{\al,\Omega}^\vee(\epsilon)))[\pi]\xrightarrow{c^{(d)}_?[\pi]^{\epsilon}} \HH^{d+1}_?(G(F),\Ah(K^\p,\Omega;V_{\al,\Omega}^\vee(\epsilon)))[\pi]$$
on $\pi$-isotypical components.

\begin{Lem}\label{smoothcup}
The map $c^{(d)}_?[\pi]^{\epsilon}$ is an isomorphism for every sign character $\epsilon$ and every degree $d$.
\end{Lem}
\begin{proof}
The proof of \cite{Ge3}, Lemma 3.8, also works in this more general setup.
\end{proof}

This together with Proposition \ref{dimensions} \eqref{thirddim} implies:
\begin{Cor}\label{fourthdim}
For every character $\epsilon\colon \pi_0(G_\infty)\to\left\{\pm 1 \right\}$ we have
$$\dim_\Omega \HH^{d+1}_?(G(F),\Ah_\Omega(K^\p,\Omega;V_{\al,\Omega}^\vee(\epsilon)))[\pi] = \binom{r_\C}{d-q}.$$
\end{Cor}




\subsection{P-adic special series}\label{Steinberg}
Throughout this section we fix a finite extension $\Omega\subseteq \overline{\Q_p}$ of $\Q_p$ such that the image of every continuous embedding $\sigma\in\Sigma_{\p}$ is contained in $\Omega$.
We write $R$ for its ring of integers.

Given an even integer $l\geq 0$ we let
$$V(l)_{\Omega}=\Sym^{l}\Omega^{2}\otimes \det^{-l/2}$$
be the algebraic representation of $PGL_{2,\Omega}$ of highest weight $l$.
We fix a tuple $k_\p=(k_\sigma)_{\sigma\in\Sigma_\p}$ of even integers $k_\sigma\geq 0$ and put
$V(k_\p)_{\Omega}=\bigotimes_{\sigma\in\Sigma_\p}V(k_\sigma)_\Omega.$
We view $V(k_\p)_\Omega$ as a $G_\p$-representation by letting it act on the $k_\sigma$-factor via the embedding $\sigma \colon G_\p\into PGL_2(\Omega)$.
Note that every irreducible $\Q_p$-rational $\Omega$-representation of $G_{F_\p}$ arises in this way.
We put $\St_\p(k_\p)_\Omega=\St_{\p,\Omega}\otimes_{\Omega}V(k_\p)_\Omega$.
\begin{Pro}\label{flawlessSteinberg}
The locally $\Q_p$-algebraic representation $\St_\p(k_\p)_\Omega$ admits a flawless $R$-lattice.
\end{Pro}
\begin{proof}
TODO: the case of $\Omega$-rational representations is \cite{Vi}, Proposition 0.9.
Same proof should work here.
But it should be somewhere in the literature.
\end{proof}

Let $B\subset G_\p$ the subgroup of upper triangular matrices.
Given a subset $J\subseteq \Sigma_\p$ and a tuple $l=(l_\sigma)_{\sigma\in \Sigma_\p}$ of integers we define the $J$-analytic character
$$\chi^J_{l}\colon B\too \Omega^{\ast},\ \begin{pmatrix} a & u \\ 0 & d \end{pmatrix} \mapstoo \prod_{\sigma\in J}\sigma(a/d)^{l_\sigma}.$$
We let $I^\prime(k_\p)^J_\Omega=\left(\Ind_{B}^{G_\p}\chi^J_{-k/2}\right)^{J-\an}$ be the locally $J$-analytic induction of the character $\chi^J_{-k/2}$ from $B$ to $G_\p$ and put
$$I(k_\p)^J_\Omega=\bigotimes_{\sigma\notin J}V(k_\sigma)_\Omega  \otimes I^\prime(k_\p)^J_\Omega.$$
Its subspace of (globally) algebraic vectors can be identified with $V(k_\p)_\Omega$.
We define
$$\St^{J-\an}_\p(k_\p ,\Omega)=I(k_\p)^{J}_{\Omega}/V(k_\p)_\Omega.$$
We have a canonical embedding
$$\St_\p(k_\p)_\Omega\too \St_\p^{J-\an}(k_\p)_\Omega.$$

\begin{Pro}\label{AmiceVelu}
Suppose that that for all $\sigma\in J$ the following bound holds:
\begin{align*}
\sum_{\tau\in \Sigma_\p,\ \tau\neq\sigma}k_\tau \leq k_\sigma.  \tag{\textasteriskcentered} \label{asteq}
\end{align*}
Then the embedding $\St_\p(k_\p)_\Omega\too \St_\p^{J-\an}(k_\p)_\Omega$ induces an isomorphism of $\Omega$-Banach representations
$$\St_\p(k_\p)_\Omega^{\univ}\too \St_\p^{J-\an}(k_\p)_\Omega^{\univ}.$$
\end{Pro}
\begin{proof}
TODO: should be somewhere in the literature (Breuil, de Ieso, Kidwell).
REMARK: this is essentially Teitelbaum's extension of Amice-Velu theory.
\end{proof}

\begin{Rem}
A standard non-criticality assumption often used in control theorems for overconvergent cohomology (see for example \cite{BWi0}, theorem 8.7) is that equation \eqref{asteq} holds for all $\sigma\in\Sigma_\p$.
This forces the prime $\p$ to be of degree one or two and the weight $(k_\sigma)_{\sigma\in\Sigma_\p}$ to be parallel. 
\end{Rem}

The following construction of extensions is due to Breuil (see \cite{Br}, Section 2.1).
Let $\lambda\colon F_\p^{\ast}\colon\Omega$ be a $J$-analytic homomorphism.
We define $\tau(\lambda)$ to be the two dimensional $\Omega$-representation given by
$$\begin{pmatrix} a & u \\ 0 & d \end{pmatrix} \mapstoo \begin{pmatrix} 1 & \lambda(a/d) \\ 0 & 1 \end{pmatrix} $$
and put $\tau^J(k_\p ,\lambda)=\tau\otimes \chi^J_{-k/2}$.
The short exact sequence
$$0 \too\chi^J_{-k/2}\too \tau^J(k_\p ,\lambda) \too \chi^J_{-k/2}\too 0$$
induces the short exact sequence
$$0 \too I\prime(k_\p)^J_{\Omega}\too \left(\Ind_{B}^{G_\p}\tau(k_\p ,\lambda)\right)^{J-\an} \too I^\prime(k_\p)^J_{\Omega}\too 0 $$
of locally $J$-analytic representations.
Tensoring with $\otimes_{\sigma\notin J}V(k_\sigma)_\Omega$ yields a self-extension of $I(k_\p)^J_{\Omega}.$
Finally, pullback via $V(k_\p)_\Omega\into I(k_\p)^J_\Omega$ and pushforward along $I(k_\p)^J_\Omega\onto \St_\p(k_\p)^{J-\an}_\Omega$ yields an exact sequence
\begin{align}\label{extension}
0 \too \St_\p^{J-\an}(k_\p)_\Omega\too \mathcal{E}^{J}(k_\p ,\lambda)_\Omega\too V(k_\p)_\Omega\too 0.
\end{align}

\begin{Rem}
Given two locally $\Q_p$-analytic $\Omega$-representations $W_1$ and $W_2$ we denote by $\Ext^{1}_{\an}(W_1,W_2)$ the space of locally $\Q_p$-analytic extensions of $W_2$ by $W_1$. The map
$$\Hom_{J-\an}(F_\p^{\ast},\Omega)\too \Ext^{1}_{\an}(\St_\p^{J-\an}(k_\p)_\Omega,V(k_\p)_\Omega),\ \lambda \mapstoo \mathcal{E}^{J}(k_\p ,\lambda)_\Omega$$
is an isomorphism.
In the case $F_\p=\Q_p$ this is due to Breuil.
In fact, an analogous statement is true for higher rank groups as well (see \cite{Ding}, Theorem 1, and \cite{Ge4}, Theorem 2.13).
\end{Rem}





\subsection{Automorphic L-invariants}\label{Definition}
Let $\Omega\subseteq \overline{\Q_p}$ be a finite extension of $\Q_p$ that contains $\Q_\pi$.
We define 
\begin{align*}
V_{\al,\p,\Omega}&=\bigotimes_{\sigma\in\Sigma_\p}V_{\sigma,\Omega}
\intertext{and}
V_{\al,\Omega}^{\p}&=\bigotimes_{\sigma\notin\Sigma_\p}V_{\sigma,\Omega}.
\end{align*}
We can extend the action of $G(F)$ on $V_{\al,\p,\Omega}$ (resp.~on $V_{\al,\Omega}^{\p}$) to an action of $G_\p$ (resp.~an action of $G_p^{\p}$).
Since $V_{\al,\p,\Omega}$ is an irreducible $\Q_p$-rational representation of $G_{F_\p}$ there exists a unique tuple $k_\p=(k_\sigma)_{\sigma\in\Sigma_\p}$ of even integers and an isomorphism $V_{\al,\p,\Omega}\cong V(k_\p)$, which is unique up to multiplication with a scalar.
We have the following chain of isomorphisms
\begin{align*}
\HH^{d}_?(G(F),\Ah_\Omega(K^\p,\St_{\p,\Omega};V_{\al,\Omega}^\vee(\epsilon)))
\xrightarrow{\ref{Example}} &\HH^{d}_?(G(F),\Ah_\Omega(K^\p,\St_{\p}(k_\p)_\Omega;(V_{\al,\Omega}^{\p})^\vee(\epsilon)))\\
\xrightarrow{\ref{automatic},\ \ref{flawlessSteinberg}} &\HH^{d}_?(G(F),\Ah_\Omega^{\cont}(K^\p,\St_{\p}(k_\p)^{\univ}_\Omega;(V_{\al,\Omega}^{\p})^\vee(\epsilon))).
\end{align*}
Let $J=J_{\mx}\subseteq\Sigma_\p$ be the maximal set of embeddings such that equation \eqref{asteq} holds for all $\sigma\in J_{\mx}$.
Given a $J$-analytic homomorphism $\lambda\colon F_\p^{\ast}\to\Omega$ we denote by
$$\mathcal{E}^{J}(k_\p ,\lambda)_\Omega\in\Ext^{1}_{\an}(\St_\p^{J-\an}(k_\p)_\Omega,V(k_\p)_\Omega)$$
be the extension associated to $\lambda$ at the end of Section \ref{Steinberg}.
By Proposition \ref{AmiceVelu} we may form the cup product
\begin{align*}&\HH^{d}_?(G(F),\Ah_\Omega^{\cont}(K^\p,\St_{\p}(k_\p)^{\univ}_\Omega;(V_{\al,\Omega}^{\p})^\vee(\epsilon)))\\
\xrightarrow{\cup \mathcal{E}^{J}(k_\p ,\lambda)_\Omega} &\HH^{d+1}_?(G(F),\Ah_\Omega(K^\p,V_{\al,\p,\Omega};(V_{\al,\Omega}^{\p})^\vee(\epsilon)))\\
\cong&\HH^{d+1}_?(G(F),\Ah_\Omega(K^\p,\Omega;(V_{\al,\Omega})^\vee(\epsilon))).
\end{align*}
Let $c^{(d)}_?(\lambda)[\pi]^{\epsilon}$ denote the restriction of this map to the $\pi$-isotypical component.

\begin{Def}
We define the $\LI$-invariant
$$\LI_?(\pi,\p)^{\epsilon}\subseteq \Hom_{{J_{\mx}-\an}}(F_\p^{\ast},\Omega)$$ of $\pi$ at $\p$ of sign $\epsilon$ as the kernel of the map $\lambda \mapsto c^{(q)}_?(\lambda)[\pi]^{\epsilon}.$
\end{Def}
Note that the $\LI$-invariant $\LI_?(\pi,\p)^{\epsilon}$ really depends on the choice of embeddings $\iota_\infty$ and $\iota_\p$ we made at the beginning.
\begin{Pro}
The following holds for every sign character $\epsilon$:
\begin{enumerate}[(a)]
\item $\LI_c(\pi,\p)^{\epsilon}=\LI(\pi,\p)^{\epsilon}$ and
\item $\LI(\pi,\p)^{\epsilon}\subseteq \Hom_{J_{\mx}-\an}(F_\p^{\ast},\Omega)$ is a subspace of codimension one that does not contain the subspace of locally constant homomorphisms.
\end{enumerate}
\end{Pro}
\begin{proof}
The first claim follows from the fact that all maps considered in the construction commute with the map $\delta$ defined in \eqref{delta}.
The second claim is a direct consequence of Proposition \ref{dimensions} and Lemma \ref{smoothcup}.
\end{proof}





\section{P-adic Hodge theory}


\section{Stark-Heegner cycles}


\bibliographystyle{alpha}
\bibliography{bibfile}


\end{document}