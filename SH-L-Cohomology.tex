\subsection{Cohomology of $\p$-arithmetic groups}\label{Cohomology}
Throughout this section we fix a ring $R$.

Let $\Div(\PP(F))$ denote the free abelian group on $\PP(F)$ and $\Div_0(\PP(F))$ the kernel of the map $$\Div(\PP(F)) \to \Z,\ \sum_P m_P P \mapsto \sum_P m_P.$$
The $PGL_2(F)$-action on $\PP(F)$ induces an action on $\Div_0(\PP(F))$.
If $G$ is non-split, we put $\HH^{i}_c(G(F),A)=\HH^{i}(G(F),A)$.
If $G$ is split, we define $\HH^{i}_c(G(F),A)=\HH^{i-1}_c(G(F),\Hom_\Z(\Div_0(\PP(F)),A)).$
In this case the boundary map associated to the short exact sequence
$$0\too A \too\Hom_\Z(\Div(\PP(F)),A)\too\Hom_\Z(\Div_0(\PP(F)),A)\too 0$$
yields a map 
\begin{align}\label{delta}
\delta\colon\HH^{i}_c(G(F),A)\too\HH^{i}(G(F),A).
\end{align}

Given a compact, open subgroup $K^\p\subseteq G(\A\pinfty)$, an $R[K^{\p}]$-module $N^{\p}$, an $R[G_\p]$-module $M_\p$ and an $R[G(F)]$-module $N$ we define $\Ah_{R}(K^{\p},N^{\p},M_\p;N)$ as the space of all $R$-bilinear maps $\Phi\colon G(\A\pinfty) \times N^{\p} \times M_\p\to N$
such that $\Phi(gk,kn,m)=k\Phi(g,n,m)$ for all $g\in  G(\A\pinfty)$, $k\in K^\p$, $n\in\N^{\p}$ and $m\in M_\p$.
The $R$-module $\Ah_R(K^\p,N^\p,M_\p;N)$ carries a natural $G(F)$-action given by
$$(\gamma.\Phi)(g,n,m)=\gamma.(\Phi(\gamma^{-1}g,n,\gamma^{-1}.m)).$$
Most of the times the module $N^{\p}$ is equal to $R$.
In this case we put $$\Ah_R(K^\p,M_\p;N)=\Ah_R(K^\p,R,M_\p;N).$$

\begin{Exa}\label{Example}
If $M_\p$ is of the form $\cind_{K_\p}^{G_\p} R$ for some compact, open subgroup $K_\p\subseteq G_\p$, we put
$$\Ah(K^\p K_\p;N)=\Ah_{R}(K^\p,M_\p;N)$$
where $?\in\left\{\emptyset,c \right\}$.
By definition we have a natural $G(F)$-equivariant isomorphism
$$\Ah(K^\p K_\p;N)\xrightarrow{\cong}C(G(\A^\infty)/K^\p K_\p,N).$$

More generally, suppose $M_\p$ is of the form $\cind_{K_\p}^{G_\p} N_\p$ for some compact, open subgroup $K_\p\subseteq G_\p$ and some $R[G_\p]$-module $N_\p$
and that $N^{\p}$ is a $G(\A\pinfty)$-module.
Then the map
$$\left(\cind_{K_\p}^{G_\p} R\right) \otimes_R N_p \too \cind_{K_\p}^{G_\p} N_\p,\ (f,n)\mapstoo [g \mapsto f(g)\cdot g.n]$$
is an isomorphism of $R[G_\p]$-modules.
Hence, its inverse (and a similar map for the $N^{\p}$-part) induces an isomorphism of $R[G(F)]$-modules
$$\Ah_{R}(K^\p,N^{\p},M_\p;N)\xrightarrow{\cong}C(G(\A^\infty)/K^\p K_\p,\Hom_R(N^{\p}\otimes_R N_\p, N)).$$
\end{Exa}

\begin{Def}
An $R[G_\p]$-module $M$ is called flawless if 
\begin{itemize}
\item $M$ is projective as an $R$-module and
\item there exists a finite length exact resolution
$$0\too P_m\too\cdots\too P_0\too M \too 0$$
of $R[G_\p]$-modules, where each $P_i$ is a finite direct sum of modules of the form
$$\cind_{K_\p}^{G_\p} L$$
with $K_\p\subseteq G_\p$ a compact, open subgroup and $L$ an $R[K_\p]$-module which is finitely generated projective over $R$.
\end{itemize}
\end{Def}

\begin{Pro}\label{FlachundNoethersch}
Suppose that $M$ is a flawless $R[G_\p]$-module and that $N^{\p}$ if finitely generated projective as an $R$-module.
For $?\in\left\{\emptyset,c \right\}$ we have:
\begin{enumerate}[(a)]
\item\label{FuN} The $R$-module $\HH^{d}_?(G(F),\Ah_{R}(K^\p,N^{\p},M_\p;N))$ is finitely generated for all $d$ if $R$ is Noetherian and $N$ is finitely generated as an $R$-module.
\item\label{FN2} If $S$ is a flat $R$-algebra, then the canonical map
	 \begin{align*}
	\HH^{d}_?(G(F),\Ah_{R}(K^\p,N^{\p},M_\p;N))\otimes_R S
	\too &\HH^{d}_?(G(F),\Ah_{S}(K^\p,N^{\p}_S, M_{\p,S};N_S))
	\end{align*}
	is an isomorphism for all $d\in\Z$.
\end{enumerate}
\end{Pro}
\begin{proof}
This is essentially Proposition 4.9 of \cite{Ge}.
\end{proof}

\begin{Exa}
Let $N$ be an $R[G(F)]$-module and $K^{\p}K_\p\subseteq G(\A^{\infty})$ a compact, open subgroup.
In light of Example \ref{Example} we put
\begin{align*}
\HH^{d}(X_{K^{\p}K_\p},N)&=\HH^{d}(G(F),\Ah(K^\p K_\p;N))
\intertext{respectively}
\HH^{d}_c(X_{K^{\p}K_\p},N)&=\HH^{d}_c(G(F),\Ah(K^\p K_\p;N)).
\end{align*}
If $K^\p K_\p$ is neat or $R$ is a field of characteristic $0$, we can identify these groups with the $N$-valued singular cohomology (respectively singular cohomology with compact support) of the locally symmetric space of level $K^\p K_\p$ associated to $G$.
\end{Exa}

Let $\Omega$ be a finite extension of $\Q_p$ with ring of integers $R$, $V_\p$ an $\Omega$-Banach representation of $G_\p$ and $V^{\p}$ a finite dimensional continuous $\Omega$-representation of $G_p^{\p}$.
We view $V^{\p}$ as a $G(F)$-representation via the embedding $G(F)\into G_p^{\p}$.
Let $\epsilon\colon \pi_0(G_\infty)\to \left\{\pm 1\right\}$ be a sign character.
We define
\begin{align*}
\Ah_\Omega^{\cont}(K^\p,V_\p;V^{\p}(\epsilon))&=C(G(\A\pinfty)/K^\p,\Hom_{\Omega,\cont}(V_\p,V^{\p}(\epsilon))).
\end{align*}

Now let $V_\p$ merely be a be a locally convex topological $\Omega$-vector space equipped with a continuous $G_\p$-action.
Suppose that $V_\p$ admits an open $R[G_\p]$-lattice $M_\p$ that is flawless.
Since $M_\p$ is finitely generated, it follows that the completion of $V_\p$ with respect to $M_\p$ is the universal unitary completion $V_\p^{\univ}$ of $V_\p$.
We have the following automatic continuity statement.
\begin{Cor}\label{automatic}
Let $V_\p$ be a finite length, locally $\Q_p$-algebraic representation of $G_\p$ that admits a $G_\p$-stable separated $R$-lattice and let $V^{\p}$ be a finite dimensional $\Omega$-representation of $G_p^{\p}$.
Then the canonical map
$$\HH^{d}_?(G(F),\Ah_{\Omega}^{\cont}(K^\p,V_\p^{\univ};V^{\p}(\epsilon)))\too \HH^{d}_?(G(F),\Ah_{\Omega}(K^\p,V_\p;V^{\p}(\epsilon)))$$
is an isomorphism for all characters $\epsilon$ and $?\in\left\{\emptyset,c\right\}$.
\end{Cor}
\begin{proof}
By \cite{Vi}, Proposition 0.4, the representation $V_\p$ admits a flawless $R$-lattice $M_\p$. 
Since $V^\p$ is finite dimensional, Example \ref{Example} implies that
$$\Ah_{\Omega}(K^\p,V_\p;V^{\p}(\epsilon))=\Ah_{\Omega}(K^\p,V^{\p,\vee},V_\p;\Omega(\epsilon)).$$
Again, by finite-dimensionality of $V^{\p,\vee}$ we see that it admits a $K_\p$-stable lattice $N_\p$.
Therefore, Proposition \ref{FlachundNoethersch} \eqref{FN2} implies that the canonical map
$$\HH^{d}_?(G(F),\Ah_{R}(K^\p,N^{\p},M_\p;R(\epsilon)))\otimes_R \Omega\too \HH^{d}_?(\Ah_{\Omega}(K^\p,V_\p;V^{\p}(\epsilon)))$$
is an isomorphism.
But the former can be identified with the cohomology group $\HH^{d}_?(G(F),\Ah_{\Omega}^{\cont}(K^\p,V_\p^{\univ};V^{\p}(\epsilon)))$
and, thus, the claim follows.
\end{proof}


