\section*{Introduction}

TODO

\smallskip
\textbf{Notations.}
All rings are assumed to be commutative and unital.
The group of invertible elements of a ring $R$ will be denoted by $R^{\ast}$.
Given an $R$-module $M$ we put $M^{\vee}=\Hom_R(M,R)$.
If $S$ is an $R$-algebra and $M$ an $R$-module, we put $M_S=M\otimes_R S.$
If $R$ is a ring and $G$ a group, we will denote the group ring of $G$ over $R$ by $R[G]$.
Let $H$ be an open subgroup of a locally profinite group $G$ and $M$ an $R$-linear representation $M$ of $H$.
The \textit{compact induction} $\cind^{G}_{H}M$ of $M$ from $H$ to $G$ is the space of all functions $f\colon G\to M$ such that:
\begin{itemize}
\item $f$ has finite support modulo $H$ and
\item $f(hg)=h.f(g)$ for all $h\in H, g\in G$.
\end{itemize}
Compact induction $\cind^{G}_{H}M$ is an $R$-module on which $G$ acts $R$-linearly via the right regular representation.
Let $\chi\colon G\to R^{\ast}$ be a character.
We write $R[\chi]$ for the $G$-representation, which underlying $R$-module is $R$ itself and on which $G$ acts via the character $\chi$.
More generally, if $M$ is any $R[G]$-module, we put $M(\chi)=M\otimes_R R(\chi).$
The trivial character will be denoted by $\cf$.

\smallskip
\textbf{Acknowledgements.}
TODO